\documentclass[a4paper,12pt]{article}
\usepackage[osf]{mathpazo}
\usepackage{ms}
\usepackage{natbib}
\usepackage{lineno}
\usepackage{graphicx}
\usepackage{caption}
\modulolinenumbers[5]
\linenumbers

\pdfminorversion=3

\makeatletter
\renewcommand{\@biblabel}[1]{\quad#1.}
\makeatother

\title{Standard principal components should not be used in phylogenetic comparative analyses}
\author{
Josef C. Uyeda$^{1,*}$, Daniel S. Caetano$^1$, and Matthew W. Pennell$^1$
}

\date{}
\affiliation{
 $^{1}$ Department of Biological Sciences \& Institute for Bioinformatics and Evolutionary Studies, University of Idaho, Moscow, ID 83844, U.S.A.\\ 
 $^{*}$ Email for correspondence: \texttt{pseudacris@gmail.com}\\
}

\runninghead{PCA in comparative analyses}
\keywords{Phylogenetic comparative methods, principal components, Brownian motion, Ornstein-Uhlenbeck}


\begin{document}

\mstitlepage
\parindent=1.5em
\addtolength{\parskip}{.3em}
\vfill

\section{Abstract}
something
\vfill

\newpage

\section{Introduction}
The units of measurements are not neutral or arbitrary; different ways of representing the same set of data can change the meaning of the measurements and alter the interpretations of subsequent statistical analyses \citep{Hand2004, HansenHoule2008, Houle2011}. Therefore, careful attention needs to be paid to how data is measured and how it can be transformed. Furthermore, this should be done with the scientific question in mind in order to ensure that resulting inferences are meaningful \citep{Houle2011}. 

For multivariate data, a common transformation is to represent the data in terms of its principal components (PC) axes. This involves converting a matrix of possibly correlated observations into a set of linear components that are by definition, orthogonal to (i.e., independent of) one another. The first PC axis is the eigenvector in the direction of the greatest variance, the second, the second greatest variance, etc. This is a very convenient representation of the data as the values for each observation for a given PC axes can be statistically analyzed independent of values for all other PCs. 

Computing PC scores prior to analyses is a common procedure in phylogenetic comparative biology. It is often done when the data is by nature, multivariate and different measures are correlated with one another. Typical examples of such data include geometric morphometric data \citep[e.g.,][]{Dornburg2011, Hunt2013}, measurements of multiple morphological traits \citep[e.g.,][]{Harmon2010, BergmannIrshick2012, Weir2012, Pienaar2013}, and climatic data \citep[e.g.,][]{KozakWiens2010, Schnitzler2012}. Phylogenetic comparative methods \citep[PCMs; recently reviewed by][]{PennellHarmon} have been developed specifically to make use of the fact that species are non-independent data points owing to shared ancestry \citep{Felsenstein1985}, and researchers from many different fields have taken an interest in these methods. While the importance of considering phylogeny is the analyses of comparative data is widely recognized, the fact that data transformations should also incorporate phylogenetic information is not.

\citet{Revell2008} developed a procedure (explained in detail below) for converting multivariate data into ``phylogenetically corrected'' PC axes, by assuming that the original data evolved along the phylogeny using a Brownian motion (BM) model of evolution. In his analysis of simulated data, he demonstrated that failing to take phylogeny into can bias the estimation of eigenvalues. Here we expand upon his argument and demonstrate that performing comparative analyses on standard PCs is positively misleading and can lead researchers to inadvertantly make biological inferences from statistical artifacts. Our point has been made in other fields that deal with autocorrelation between observations, such as population genetics \citep{Novembre}, ecology \citep{Podani2002}, climate research \citep{Richman1986} and paleobiology \citep{Bookstein2012}. However, the connection between these previous results and phylogenetic comparative data has not been explicitly made and standard PCs continue to be widely used in the field. Furthermore, \citet{Revell2008} used a BM model to calculate the phylogenetically--corrected PC axes; it is unclear whether this is a valid procedure when alternate models (e.g., Ornstein-Uhlenbeck) are fit to the data. We argue that standard PC axes should never be analyzed in a phylogenetic comparative biology and point to alternatives that have been developed previously and to future directions for fully multivariate comparative methods.

\section{Methods}
\subsection{Computing PC scores}

\begin{equation}
\mathbf{a}=[(\mathbf{1}^\prime \mathbf{C}^{-1} \mathbf{1})^{-1} 
\mathbf{1}^\prime \mathbf{C}^{-1} \mathbf{X}]^\prime
\end{equation}

\begin{equation}
\mathbf{R} = (n-1)^{-1} (\mathbf{X} - \mathbf{1a}^\prime ) \mathbf{C}^{-1} 
(\mathbf{X} - \mathbf{1a}^\prime )
\end{equation}

\subsection{Simulations}

\section{Results}

\section{Discussion}

Discussion of main results

\subsection{Phylogenetic PCA for non-Brownian models}
Does PhyloPCA works for non brownian models

\subsection{Alternatives to PCA}
Discuss developments of multivariate comparative methods with a focus on interpretability

\section{Concluding remarks}
Drive the point home

\bibliographystyle{jecol}
\bibliography{phylopca.bib}
\end{document}
